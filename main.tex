\documentclass[12pt,a4paper]{article}
\usepackage[T2A]{fontenc}
\usepackage[utf8]{inputenc}
\usepackage[russian]{babel}
\usepackage{amsmath}
\usepackage{amsfonts}
\usepackage{amssymb}
\usepackage{graphicx}
\usepackage{geometry}


\geometry{
a4paper,
total={170mm,257mm},
left=20mm,
top=20mm,
}
\title{Анализ существующих методов и алгоритмов анализа программных систем, использующих системы контроля версий}
\author{Перепелицына Екатерина ПИмд-11}
\date{}

\begin{document}
\maketitle

\section{Введение}

Система контроля версий(СКВ)—это система, записывающая изменения в файл или набор файлов в течение времени и позволяющая вернуться позже к определённой версии. 
Анализ проектов, использующих СКВ, позволяет отслеживать изменения проекта в любой момент времени. Например, анализ подключаемых библиотек к проекту можно сделать по последнему коммиту, а можно отследить все подключаемые к проекту библиотеки, используя СКВ, переключаясь между коммитами и получить некоторую динамику.
Исследование существующих методов и алгоритмов анализа любых изменений и действий через СКВ помогает лучше понять, как использовать СКВ для получения максимальной информации о проекте, на какие изменения стоит обращать внимание и какие данные было бы интересно отслеживать.

\section{Публикации}

В статье \cite{Article1} приведен анализ характера изменений программ и поиск неисправленных фрагментов кода. Целью данной работы является разработка методов анализа характера изменений между версиями компонентов ПО, для которых отсутствует исходный код.

В статье \cite{Article2} описан прототип автоматизированной системы, основной задачей которой является поиск и подбор команд специалистов на основе данных открытых репозиториев исходного кода и связанных артефактов. В статье подробно рассматриваются состав архитектуры системы, алгоритм выбора основной команды проекта, выявленные в ходе исследования метрики деятельности группы, формулы расчетов значений метрик, а также их применение при решении задачи анализа проектного репозитория.

В статье \cite{Article3} рассмотрена архитектура системы обработки больших данных, основанной на инструментах Apache Hadoop, Apache Flume и Apache Spark. Продемонстрировано применение разработанной системы для хранения и анализа наборов данных, состоящих из генерируемых событий в репозитории GitHub – крупнейшего в мире веб-сервиса на базе системы контроля версий Git.

В статье \cite{Article4} рассматриваются вопросы, связанные с использованием методов анализа данных применительно к программным репозиториям. В работе делается попытка представить обзор технологий, которые используются при анализе программ и базируются на статических данных, которые могут быть извлечены непосредственно из программного кода или репозиториев кода. В работе приводится обзор работ, использующих методы глубокого обучения (рекуррентные нейронные сети), методы классификации, основанные на других моделях машинного обучения, а также использование кластеризации в программной инженерии. Практические области применения рассматриваемых методов включают в себя, например, классификацию и предсказание ошибок, определение характеристики изменения кода во времени, поиск дублирующих фрагментов, автоматическое обнаружение ошибок проектирования, выдачу рекомендаций по рефакторингу кода.

В ВКР \cite{Article5} рассматривается разработка системы отслеживания ошибок в программных продуктах. Проведен анализ процесса тестирования программных продуктов на наличие ошибок, описывается процесс создания системы отслеживания ошибок в программных продуктах.

В статье \cite{Article6} рассматривается возможность прогнозирования тенденций репозитория и языков программирования с использованием временных рядов и анализом событий репозиториев GitHub.

В статье \cite{Article7} приводят результаты анализа тональности текста коммитов, так как эмоции имеют высокое влияние на продуктивность, качество выполнения задач и прочее. В основном сообщения к коммитам являются нейтральными, но бывают исключения. В исследовании были рассмотрены проекты на 14 разных языках программирования, самые негативные комментарии к коммитам наблюдаются в проектах, написанных на языке программирования Java, также самые негативные комментарии были оставлены в понедельники.

В статье \cite{Article8} рассматривают анализ репозиториев и пользователей GitHub. Рассматриваются зависимости между поведением пользователя и успешности проектов, в которых он участвует. В статье пытаются выявить закономерности, понять, что имеет наибольшее влияние на успешность проекта для возможности дальнейшего прогноза.

В статье \cite{Article9} рассматривается получение информации с сайта GitHub. Так как Github предоставляет API для доступа к большому количеству информации, многие пытаются получить какую-то полезную информацию с этого ресурса. В статье приводятся основные ошибки, допускаемые при анализе данных, получаемых при неправильных запросах и как лучше строить свои запросы к сервису.


В статье \cite{Article10} предложен к рассмотрению инструмент, созданный для получений структурированных данных по ряду параметров, заданных в запросе к API Github. Инструмент предлагает интерфейс для создания запросов более удобный, чем прямое обращение к API.

\section*{Заключение}
В результате анализа имеющихся алгоритмов и результатов анализа предыдущих исследователей были рассмотрены разработанные инструменты,алгоритмы и методы анализа и получения данных с систем контроля версий, какие данные можно получить и как ее можно анализировать.


\begin{thebibliography}{10}
\bibitem{Article1}АРУТЮНЯН М.С., ИВАНОВ Г.С., ВАРДАНЯН В.Г., АСЛАНЯН А.К., АВЕТИСЯН А.И., КУРМАНГАЛЕЕВ Ш.Ф. АНАЛИЗ ХАРАКТЕРА ИЗМЕНЕНИЙ ПРОГРАММ И ПОИСК НЕИСПРАВЛЕННЫХ ФРАГМЕНТОВ КОДА. [Электронный ресурс] - URL:https://elibrary.ru/item.asp?id=37313183

\bibitem{Article2}ЯРУШКИНА НАДЕЖДА ГЛЕБОВНА, ЖЕЛЕПОВ АЛЕКСЕЙ СЕРГЕЕВИЧ ПРОТОТИП СИСТЕМЫ ПОИСКА И ВЫБОРА "СФОРМИРОВАННЫХ" КОМАНД ИТ-СПЕЦИАЛИСТОВ НА ОСНОВЕ ДАННЫХ ПРОЕКТНЫХ РЕПОЗИТОРИЕВ. [Электронный ресурс] - URL:https://elibrary.ru/item.asp?id=42665427

\bibitem{Article3}ВОИНОВ Н.В., K. РОДРИГЕС ГАРСОН, НИКИФОРОВ И.В., ДРОБИНЦЕВ П.Д. СИСТЕМА ОБРАБОТКИ БОЛЬШИХ ДАННЫХ ДЛЯ АНАЛИЗА СОБЫТИЙ РЕПОЗИТОРИЯ GITHUB. [Электронный ресурс] - URL:https://elibrary.ru/item.asp?id=38309155

\bibitem{Article4}Д.Е. Намиот, В.Ю. Романов Анализ данных для программных репозиториев [Электронный ресурс] - URL:http://injoit.org/index.php/j1/article/view/560

\bibitem{Article5}Яцутко, Сергей Анатольевич Разработка системы отслеживания ошибок в программных продуктах. [Электронный ресурс] - URL:http://elib.sfu-kras.ru/handle/2311/141513

\bibitem{Article6}T V Varuna; Anuraj Mohan Trend Prediction of GitHub using Time Series Analysis [Электронный ресурс] - URL:https://ieeexplore.ieee.org/abstract/document/8944878

\bibitem{Article7}Emitza Guzman, David Azócar, Yang Li Sentiment Analysis of Commit Comments in GitHub: AnEmpirical Study. [Электронный ресурс] - URL:https://www.researchgate.net/publication/266657943_Sentiment_analysis_of_commit_comments_in_GitHub_An_empirical_study

\bibitem{Article8}Fragkiskos Chatziasimidis; Ioannis Stamelos Data collection and analysis of GitHub repositories and users. [Электронный ресурс] - URL:https://ieeexplore.ieee.org/abstract/document/7388026

\bibitem{Article9}Georgios Gousios; Diomidis Spinellis Mining Software Engineering Data from GitHub. [Электронный ресурс] - URL:https://ieeexplore.ieee.org/document/7965403

\bibitem{Article10}Shreyansh Surana, Smit Detroja, Saurabh Tiwari A Tool to Extract Structured Data from GitHub. [Электронный ресурс] - URL:https://arxiv.org/abs/2012.03453

\end{thebibliography}

\end{document}