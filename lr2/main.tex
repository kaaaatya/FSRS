\documentclass[t]{beamer}

\usetheme{HSE}

%%% Работа с русским языком
\usepackage{cmap}					% поиск в PDF
\usepackage{mathtext} 				% русские буквы в формулах
\usepackage[T2A]{fontenc}			% кодировка
\usepackage[utf8]{inputenc}			% кодировка исходного текста
\usepackage[english,russian]{babel}	% локализация и переносы

%%% Работа с картинками
\usepackage{graphicx}  % Для вставки рисунков
\graphicspath{{images/}{images2/}}  % папки с картинками
\setlength\fboxsep{3pt} % Отступ рамки \fbox{} от рисунка
\setlength\fboxrule{1pt} % Толщина линий рамки \fbox{}
\usepackage{wrapfig} % Обтекание рисунков текстом

%%% Работа с таблицами
\usepackage{array,tabularx,tabulary,booktabs} % Дополнительная работа с таблицами
\usepackage{longtable}  % Длинные таблицы
\usepackage{multirow} % Слияние строк в таблице

%%% Программирование
\usepackage{etoolbox} % логические операторы

%%% Другие пакеты
\usepackage{lastpage} % Узнать, сколько всего страниц в документе.
\usepackage{soul} % Модификаторы начертания
\usepackage{csquotes} % Еще инструменты для ссылок
%\usepackage[style=authoryear,maxcitenames=2,backend=biber,sorting=nty]{biblatex}
\usepackage{multicol} % Несколько колонок

%%% Картинки
\usepackage{tikz} % Работа с графикой
\usepackage{pgfplots}
\usepackage{pgfplotstable}

\title{Разработка и исследование методов и алгоритмов анализа динамики предпочтений выбора компонентов программных систем}
\author{Перепелицына Екатерина}
\date{\today}
\institute{УлГТУ}

\begin{document}

\frame[plain]{\titlepage}	% Титульный слайд

\section{Цель}
 
\begin{frame}
		\frametitle{Цель}
		\begin{quotation}
			Целью работы является разработка методов и алгоритмов извлечения и анализа информации о подключаемых к проектам библиотеках из программных репозиториев.
		\end{quotation}
	\end{frame}
	
	\begin{frame}
		\frametitle{Объект и предмет исследования}
		\begin{quotation}
			Объектом исследования являются открытые программные репозитории из системы контроля версий Github 
		\end{quotation}
		
		\begin{quotation}
			Предметом исследования является информация о подключаемых к проектам библиотеках
		\end{quotation}
	\end{frame}

	\begin{frame}
		\frametitle{Задачи}
		\begin{itemize}
		    \item Выбор характеристик отбора программных репозиториев
			\item Отбор программных репозиториев для формирования исходных данных для анализа по выбранным характеристикам
			\item Выбор данных из репозиториев, необходимых для сохранения в бд
			\item Извлечение информации о подключаемых библиотеках не зависимо от используемых систем сборки проектов
			\item Анализ собранных данных
		\end{itemize}
	\end{frame}	
	
	\begin{frame}
		\frametitle{Научная новизна}
		\begin{quotation}
			Научная новизна на данный момент не определена
		\end{quotation}
	\end{frame}

	\begin{frame}
	    \begin{quotation}
			Исходными данными системы являются программные репозитории, отобранные по ряду параметров и полученные с помощью библиотеки Octokit через Github API
		\end{quotation}
		\frametitle{Исходные данные системы}
		\begin{figure}
        \includegraphics[width=150pt,height=120pt]{Octokit.png}
        \caption{Библиотека Octokit}
		\end{figure}
	\end{frame}

	\begin{frame} 
		\frametitle{Положение выносимые на защиту}
		\begin{itemize}
			\item алгоритм извлечения информации о подключаемых библиотеках не зависимо от используемых систем сборки проектов
			\item алгоритм прогноза популярности библиотек
		\end{itemize}
	\end{frame}
	
	\begin{frame} 
		\frametitle{Обзор научных работ по схожей тематике}
		\begin{itemize}
		    \item Система обработки больших данных для анализа событий репозиторий GITHUB. Воинов Н.В., K. Родригес Гарсон, Никифоров И.В., Дробинцев П.Д.
			\item Анализ данных для программных репозиториев. Д.Е. Намиот, В.Ю. Романов
			\item Trend Prediction of GitHub using Time Series Analysis. T V Varuna, Anuraj Mohan
		\end{itemize}
	\end{frame}

\end{document}